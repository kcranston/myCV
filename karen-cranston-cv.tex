\documentclass[10pt]{article}
\usepackage{geometry}                % See geometry.pdf to learn the layout options. There are lots.
\geometry{letterpaper}                   % ... or a4paper or a5paper or ... 
\usepackage{graphicx}
\usepackage[hyphens]{url}
\usepackage{fancyhdr}
\usepackage{hyperref}
\usepackage{parskip}
\usepackage{helvet}
\usepackage[all]{nowidow}

\renewcommand{\familydefault}{\sfdefault}
\renewcommand{\arraystretch}{2}
\hypersetup{backref,  urlcolor=blue, colorlinks=true, citecolor=black, linkcolor=black, hyperindex=true}

\title{Brief Article}
\author{The Author}
\pagestyle{fancy}
\lhead{\bf \large Karen A. Cranston} 

\begin{document}

Biological Informatics Centre of Excellence  \\
Agriculture and Agri-food Canada \\
Ottawa, ON, Canada \\
karen.cranston@gmail.com \\
Twitter: \href{https://twitter.com/kcranstn}{@kcranstn} \\
ORCID: \href{http://orcid.org/0000-0002-4798-9499}{0000-0002-4798-9499} \\
GitHub: \href{https://github.com/kcranston}{kcranston}

\section*{Education}
\begin{table}[h]
	\begin{tabular}{ p{2.0cm} p{12.4cm} }
		2002-2007 & {\bf PhD}, Department of Medical Genetics, University of Alberta, Edmonton, AB, Canada. Thesis title: ``Inferring, testing and summarizing a posterior distribution of phylogenies''. Supervisor: Dr. Bruce Rannala. \\
		1992-1996 & {\bf B.Sc. (Honours, First Class)}, Genetics. University of Manitoba, Winnipeg, MB, Canada \\
%		1995-1996 & \textbf{Undergraduate Honours project}, ``DNA sequence analysis of a antisense gene pair'', Department of Microbiology, University of Manitoba, Winnipeg, MB. \\
	\end{tabular}
\end{table}

% Employment
\section*{Employment}

2017-present:  \textbf{Technical Advisor, Bioinformatics}, Biological Informatics Centre of Excellence, Agriculture and Agri-food Canada, Ottawa, ON \\

2015-2017: \textbf{Research Scientist}, Department of Biology, Duke University, Durham, NC \\

2010-2015: \textbf{Training Coordinator and Bioinformatics Project Manager}, National Evolutionary Synthesis Center (NESCent), Durham, NC \\

2009-2010: \textbf{Postdoctoral Research Associate}, Biodiversity Synthesis Center / Encyclopedia of Life at the Field Museum of Natural History, Chicago, IL, USA  \\

2007-2009: \textbf{Postdoctoral Research Associate}, Department of Ecology and Evolution, University of Arizona, Tucson, Arizona, USA \\

1997-2002: \textbf{Laboratory technician} in cell culture, protein chemistry and analytical method validation, Cangene Corporation, Winnipeg, MB  \\	
	
1996-1997: \textbf{Technical writer} Integrated Engineering Software, Winnipeg, MB.   \\		

%Pubs
\section*{Peer-reviewed publications}

Stoltzfus, A., Rosenberg, M., Lapp, H., Budd, A., Cranston, K., Pontelli, E., ... \& Vos, R. A. (2017). Community and Code: Nine Lessons from Nine NESCent Hackathons. F1000Research, 6. \url{http://dx.doi.org/10.12688/f1000research.11429.1}

McTavish, E. J., Drew, B. T., Redelings, B., \& Cranston, K. A. (2017). How and Why to Build a Unified Tree of Life. BioEssays, 39(11). \url{https://doi.org/10.1002/bies.201700114}

Wilson, G., et al., (2017). Good Enough Practices in Scientific Computing. PLoS Comput Biol 13(6): e1005510 \url{https://doi.org/10.1371/journal.pcbi.1005510}

Rees, J., \& Cranston, K. (2017). Automated assembly of a reference taxonomy for phylogenetic data synthesis. Biodiversity Data Journal, 5, e12581. \url{https://doi.org/10.3897/BDJ.5.e12581}

McTavish, E. J., Hinchliff, C. E., Allman, J. F., Brown, J. W., Cranston, K. A., Holder, M. T., ... \& Smith, S. A. (2015). Phylesystem: a git-based data store for community curated phylogenetic estimates. Bioinformatics, btv276. \url{http://dx.doi.org/10.1093/bioinformatics/btv276} 

Hinchliff, C. E., Smith, S. A., Allman, J. F., Burleigh, J. G., Chaudhary, R., Coghill, L. M., ... \& Cranston, K. A. (2015). Synthesis of phylogeny and taxonomy into a comprehensive tree of life. Proceedings of the National Academy of Sciences, 112(41), 12764-12769. \url{http://dx.doi.org/10.1073/pnas.1423041112}. \textit{In the Altmetrics Top 100 for 2015 and in top 5\% overall; 53 news reports; 832 Tweets, 14 blogs; https://www.altmetric.com/details/4529894}

Ksepka, D.T., Parham, J.F., Allman, J.F., Benton, M.J., Carrano, M.T., Cranston, K.A., ... \& Warnock, R.C.M. (2015) The Fossil Calibration Database, A New Resource for Divergence Dating. \textit{Systematic Biology}. \url{http://dx.doi.org/10.1093/sysbio/syv025}

Teal, T. K., Cranston, K. A., Lapp, H., White, E., Wilson, G., Ram, K., Pawlik, A. (2015) Data Carpentry: Workshops to Increase Data Literacy for Researchers. \textit{International Journal of Digital Curation}. 10(1):135-143. \url{http://dx.doi.org/10.2218/ijdc.v10i1.351}
 
Cranston, K. A., Harmon, L. J., O'Leary, M. A., Lisle, C. (2014) Best Practices for Data Sharing in Phylogenetic Research. \textit{PLOS Currents Tree of Life}. Jun 19. Edition 1. \url{http://dx.doi.org/10.1371/currents.tol.bf01eff4a6b60ca4825c69293dc59645} 

Stoltzfus, A., Lapp, H., Matasci, N., Deus, H., Sidlauskas, B., Zmasek, C. M., ... \& Jordan, G. (2013). Phylotastic! Making tree-of-life knowledge accessible, reusable and convenient. \textit{BMC bioinformatics}, 14(1), 158. \url{http://dx.doi.org/10.1186/1471-2105-14-158} 

Rodrigo, A., Alberts, S., Cranston, K., Kingsolver, J., Lapp, H., McClain, C., ... \& Wiegmann, B. (2013). Science incubators: synthesis centers and their role in the research ecosystem. PloS biology, 11(1), e1001468. \url{http://dx.doi.org/10.1371/journal.pbio.1001468} 

Goff, S. A., Vaughn, M., McKay, S., Lyons, E., Stapleton, A. E., Gessler, D., ... \& Kleibenstein, D. J. (2011). The iPlant collaborative: cyberinfrastructure for plant biology. \textit{Frontiers in plant science}, 2. \url{http://dx.doi.org/10.3389/fpls.2011.0003} 

Evans, M. E., Hearn, D. J., Theiss, K. E., Cranston, K., Holsinger, K. E., \& Donoghue, M. J. (2011). Extreme environments select for reproductive assurance: evidence from evening primroses (Oenothera). \textit{New Phytologist}, 191(2), 555-563. \url{http://dx.doi.org/10.1111/j.1469-8137.2011.03697.x} 

Ammiraju, J. S., Fan, C., Yu, Y., Song, X., Cranston, K. A., Pontaroli, A. C., ... \& Wing, R. A. (2010). Spatiotemporal patterns of genome evolution in allotetraploid species of the genus Oryza. \textit{The Plant Journal}, 63(3), 430-442. \url{http://dx.doi.org/10.1111/j.1365-313X.2010.04251.x} 

Cranston, K. A. (2010) Quantifying gene tree incongruence across varying phylogenetic depths. In L. Lacey Knowles, Laura S. Kubatko (Eds.) \textit{Estimating Species Trees: Practical and Theoretical Aspects}, Wiley-Blackwell.

Cranston, K. A., Hurwitz, B., Sanderson, M. J., Ware, D., Wing, R. A., \& Stein, L. (2010). Phylogenomic analysis of BAC-end sequence libraries in Oryza (Poaceae). Systematic botany, 35(3), 512-523. \url{http://dx.doi.org/10.1600/036364410792495872} 

Cranston, K. A., Hurwitz, B., Ware, D., Stein, L., \& Wing, R. A. (2009). Species trees from highly incongruent gene trees in rice.  \textit{Systematic Biology} 58(5), 489-500. \url{http://dx.doi.org/10.1093/sysbio/syp054} 

Sanderson, M. J., Boss, D., Chen, D., Cranston, K. A., \& Wehe, A. (2008). The PhyLoTA Browser: processing GenBank for molecular phylogenetics research. \textit{Systematic Biology} 57(3), pp. 335-346. \url{http://dx.doi.org/10.1080/10635150802158688} 

Cranston, K. A., \& Rannala, B. (2007). Summarizing a posterior distribution of trees using agreement subtrees. \textit{Systematic Biology} 56(4), pp. 578-590. \\
\url{http://dx.doi.org/10.1080/10635150701485091} 

Cranston, K., \& Rannala, B. (2005). Closing the gap between rocks and clocks. Heredity, 94(5), 461-462. \url{http://dx.doi.org/10.1038/sj.hdy.6800644}

%In-prep
\section*{Commentaries, preprints, other non-peer reviewed}

(published preprints not shown)

Katz, D.S., Choi, S.T., Wilkins-Diehr, N., Hong, N.C., Venters, C.C., ... \& Littauer, R. Report on the Second Workshop on Sustainable Software for Science: Practice and Experiences (WSSSPE2). arXiv preprint \href{http://arxiv.org/abs/1507.01715}{arXiv:1507.01715}

Katz, D. S., Allen, G., Hong, N. C., Cranston, K., Parashar, M., Proctor, D., ... \& Wilkins-Diehr, N. (2014). Second Workshop on on Sustainable Software for Science: Practice and Experiences (WSSSPE2): Submission, Peer-Review and Sorting Process, and Results. arXiv preprint \href{http://arxiv.org/abs/1411.3464}{arXiv:1411.3464}.

Vision, T., \& Cranston, K. A. (2014) Open data for evolutionary synthesis: an introduction to the NESCent collection. Nature Scientific Data. \href{http://dx.doi.org/10.1038/sdata.2014.30}{doi:10.1038/sdata.2014.30}

Cranston, K. A., Blackburn, D., Brown, J. W., Dececchi, A., Gardner, N., Greshake, B., ... \& Wolfe, J. (2014): Simple rules for sharing phylogenetic data. figshare. \href{http://dx.doi.org/10.6084/m9.figshare.997763}{doi:10.6084/m9.figshare.997763}

Rees, J. A., Cranston, K. A., Lapp, H., \& Vision, T. (2013): Response to GBIF request for consultation on data licenses. figshare. \href{http://dx.doi.org/10.6084/m9.figshare.799766}{doi:10.6084/m9.figshare.799766}

Cranston, K. A., Vision, T., O'Meara, B., \& Lapp, H. (2013): A grassroots approach to software sustainability. figshare.
\href{http://dx.doi.org/10.6084/m9.figshare.790739}{doi:10.6084/m9.figshare.790739}

% Invited talks
\section*{Invited talks}
\begin{itemize}
\item{\textit{Synthesizing phylogeny and taxonomy into a comprehensive tree of life}, Biodiversity Informatics (TDWG) (2014)}
\item{\textit{Data and services for the tree of life}, Commonwealth Scientific and Industrial Research Organisation (CSIRO) (2014)}
\item{\textit{Technical and social challenges in synthesizing the tree of life}, Agriculture Canada (2014)}
\item{\textit{Enabling science with the tree of life}, Carleton University (2014)}
\item{\textit{Technical and social challenges in synthesizing the tree of life}, \url{http://phyloseminar.org} (2014)}
\item{\textit{Open Tree of Life: Synthesizing phylogenetic data into a tree of all life}, University of North Carolina Charlotte (2013)}
\item{\textit{Phylogenetics, phylogenomics and phyloinformatics: Finding evolutionary signals in large data sets'}, University of Toronto (2010)}
\item{\textit{Automating phylogenetic tree inference from public sequence databases}, University of Missouri, 2009} 
\item{\textit{Automating phylogenetic tree inference from public sequence databases}, Botany meetings, 2009} 
\item{\textit{Species trees and gene trees from high-throughput sequence data in rice}, University of Michigan, 2009}
\item{\textit{A species tree from 2000 gene trees in rice (Oryza)}, Evolution, 2008}
\item{\textit{The PhyLoTA browser: Processing GenBank for molecular phylogenetic research}, George Washington University, 2008}
\item{\textit{Species trees from gene trees in rice: When 2.5 million characters are not enough...}, University of California Davis, 2008}
\end{itemize}
%\vskip 4mm

% Conferences
%\section*{Conference presentations}

%\begin{itemize}
%\item{``Open Tree of Life: An automated and Community Assembled tree of life'', Evolution (2012)}
%\item{``Making the tree of life Phylotastic'', iEvoBio (2012)}
%\item{``Visualizing Phylogenies and Metadata at the Scale of One Million Tips'', iEvoBio (2010)}
%\item{``Phylogenetic inference strategies for thousands of gene trees'', Evolution (2010)}
%\item{``The PhyLoTA browser: Hierarchies of phylogenetic content from GenBank'', Evolution (2009)}
%\item{``Rice as a model system for phylogenomics'', Society for Molecular Biology and Evolution (2008)}
%\item{``Summarizing a posterior distribution of phylogenies'', Society for Molecular Biology and Evolution (2006)}
%\item{``Detecting convergence in Bayesian phylogenetic inference'', Evolution (2004)}
%\item{``A novel proposal mechanism for Bayesian phylogenetic inference'', Society for Molecular Biology and Evolution (2003)}
%\end{itemize}
%\vskip 4mm

%\newpage
% awards
\section*{Grants and Awards}
\begin{itemize}
\item{NSF Special Creativity Award: Collaborative Research: Automated and community-driven synthesis of the tree of life. NSF AVAToL (2014) \$1.4 million (\$1.1 million to Duke, PI Cranston, 2015-2017)}
\item{NSF AVAToL: Collaborative Research: Automated and community-driven synthesis of the tree of life. \$5.76 million (\$927,000 to Duke, PI Cranston, 2012-2015)}
\item{The iPlant Collaborative; a cyberinfrastructure-centered community for a new plant biology - subcontract to Field Museum of Natural History, (\$62000, PI Cranston, 2009-2010) and to Duke University (\$14500, PI Vision, 2011-2011)}
\item{Dissertation Fellowship, University of Alberta (2006)}
\item{Province of Alberta Graduate Student Fellowship (2006)}
\item{Mary Louise Imrie Graduate Student Award, University of Alberta (2006)}
\item{75th Anniversary Graduate Student Award, Faculty of Medicine and Dentistry, University of Alberta (2004)}
\item{Medical Sciences Graduate Research Assistantship, Faculty of Medicine and Dentistry, University of Alberta (2006 \& 2004)}
\end{itemize}
%\vskip 4mm

% Other 'lookin good' stuff
\section*{Professional activities}
\begin{itemize}
\item{Board of Directors: Open Bioinformatics Foundation (\url{https://open-bio.org/}), 2015-present}
\item{Co-organizer: Reproducible science hackathon (\url{https://github.com/Reproducible-Science-Curriculum/Reproducible-Science-Hackathon-Dec-08-2014}), 2014}
\item{Data Carpentry founding board member (\url{http://datacarpentry.org}), 2014-present}
\item{Organization administrator for NESCent Google Summer of Code participation, 2011-2013}
\item{Program committee, Workshop on Sustainable Software for Science: Practice and Experiences (WSSSPE), \url{http://wssspe.researchcomputing.org.uk}, 2014 and 2016}
\item{Panelist, First Workshop on Sustainable Software for Science: Practice and Experiences (WSSSPE), \url{http://wssspe.researchcomputing.org.uk}, 2013}
\item{Society of Systematic Biology council, 2012-2014}
\item{Systematic Biology Editorial Board, 2011-present}
\item{Leadership team, Informatics for Evolutionary Biology (iEvoBio, \url{http://ievobio.org}) conference, 2013-present}
\item{Leadership team, Hackathons, Interoperability and Phylogenetics (HIP) NESCent working group, 2011-present}
\item{Phyloinformatics Research Foundation board member, 2010-present}
\item{Team lead, iPlant Tree Of Life working group on Tree Visualization, 2009-2012}
\item{Co-organizer: Phyloinformatics VoCamp for development of an ontology for evolutionary biology, November 2009, Montpellier, France}
\item{Invited participant: National Evolutionary Synthesis Center (NESCent) Hackathon on Evolutionary Database Interoperability, March 2009, Durham, NC, USA}
\item{Invited participant: iPlant Collaborative (\url{http://iplantcollaborative.org}) Grand Challenge Workshop on Assembling the Tree of Life to Enable the Plant Sciences, November 2009, Biosphere 2, Oracle, AZ}
\item{Journal referee for: Systematic Biology, Trends Ecology and Evolution, PeerJ, Nucleic Acids Research, Journal of Open Research Software, PLOS Computational Biology, Molecular Biology and Evolution}. See \href{https://publons.com/a/213683/}{Publons reviewer profile} for recent activity. 
\item{National Science Foundation panelist, 2011 - 2014}
%\item{Reviews for ``Computational Molecular Evolution" by Z. Yang, 2007}
%\item{Professional Memberships: Society of Systematic Biology, Society for Molecular Biology and Evolution, Botanical Society of America, Biodiversity Information Standards (TDWG)}
\end{itemize}

% Teaching
\section*{Teaching}
\begin{itemize}
\item{Coordinator: NESCent Academy (\url{http://academy.nescent.org}): a series of hands-on summer workshops in Evolutionary Biology and Informatics, 2010-2014.}
\item{Certified Instructor: Software and Data Carpentry (\url{http://software-carpentry.org} \& \url{http://datacarpentry.org}), 2013-present}
\item{Instructor: Computational biology workshops, Wuhan Institute of Virology and Fudan University, China, 2013}
\item{Instructor: Phylogenetics workshops, Kenya Medical Research Institute, 2012}
\item{Instructor: Phylogenetics workshop, Research Experience for Undergraduates (REU) Program, Field Museum of Natural History, 2010}
\item{Instructor: Workshop in applied phylogenetics, Bodega Bay Marine Laboratory, 2006}
\end{itemize}
%\vskip 4mm

\end{document}  
