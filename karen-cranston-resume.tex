\documentclass[10pt]{article}
\usepackage{geometry}                % See geometry.pdf to learn the layout options. There are lots.
\geometry{letterpaper}                   % ... or a4paper or a5paper or ... 
\usepackage{graphicx}
\usepackage{amssymb}
\usepackage{epstopdf}
%\usepackage{cv}
\usepackage{url}
\usepackage{fancyhdr}
\renewcommand{\familydefault}{\sfdefault}
\renewcommand{\arraystretch}{2}

\title{Brief Article}
\author{The Author}
%\date{}                                           % Activate to display a given date or no date
\pagestyle{fancy}
\lhead{\bf \large Karen A. Cranston} 
\setlength{\parindent}{10 pt}

\begin{document}
\noindent National Evolutionary Synthesis Center \hfill @kcranstn \\
2024 W. Main Street, Suite A200  \hfill http://slideshare.net/kcranstn \\
Durham, NC, USA 27705 \hfill http://github.com/kcranston \\
karen.cranston@nescent.org  \\

\vspace{0.4cm}
I am an evolutionary computational biologist who is passionate about how computational tools and skills facilitate the adoption and practice of open science. My current home is the National Evolutionary Synthesis Center (NESCent, http://nescent.org), where I manage informatics support for a broad range of evolutionary science projects, run training programs in informatics and analytical skills for scientists, and help grow the wider community of open-source developers in biodiversity and evolutionary informatics. My current research project is Open Tree of Life (\url{http://opentreeoflife.org}), which aims to produce the first complete tree of life from synthesis of existing published phylogenies. 

\section*{Education}
\begin{table}[h]
	\begin{tabular}{ p{2.0cm} p{12.4cm} }
		2002-2007 & {\bf PhD}, Department of Medical Genetics, University of Alberta, Edmonton, AB, Canada. Thesis title: ``Inferring, testing and summarizing a posterior distribution of phylogenies''. Supervisor: Dr. Bruce Rannala. \\
		1992-1996 & {\bf B.Sc. (Honours, First Class)}, Genetics. University of Manitoba, Winnipeg, MB, Canada \\
%		1995-1996 & \textbf{Undergraduate Honours project}, ``DNA sequence analysis of a antisense gene pair'', Department of Microbiology, University of Manitoba, Winnipeg, MB. \\
	\end{tabular}
\end{table}

% Education
\section*{Employment}
\begin{table}[h]
	\begin{tabular}{ p{2.0cm} p{12.4cm} }
		2010-present & \textbf{Training Coordinator and Bioinformatics Project Manager}, National Evolutionary Synthesis Center (NESCent), Durham, NC \\
		2009-2010 & \textbf{Postdoctoral Research Associate}, Biodiversity Synthesis Center / Encyclopedia of Life at the Field Museum of Natural History, Chicago, IL  \\
		2007-2009 & \textbf{Postdoctoral Research Associate}, Department of Ecology and Evolutionary Biology, University of Arizona, Tucson, AZ \\
		1997-2002 & \textbf{Laboratory technician}, Cangene Corporation, Winnipeg, MB. Worked in Cell culture, Protein chemistry and Analytical method validation. \\
		1996-1997 & \textbf{Technical writer}, Integrated Engineering Software, Winnipeg, Canada, 1996-1997. Wrote technical documentation and online help systems for engineering simulation software. \\
	\end{tabular}
\end{table}

%Pubs
%\vspace{2mm}
\section*{Publications}

\noindent{\sc Arlin Stoltzfus, Hilmar Lapp, Naim Matasci, et al.} (2013) Phylotastic! Making tree-of-life knowledge accessible, reusable and convenient. \textit{BMC Bioinformatics}, 14:158. \url{doi:10.1186/1471-2105-14-158} \\

\noindent{\sc Allen Rodrigo, Susan Alberts, Karen Cranston, et al.} (2013) Science Incubators: Synthesis Centers and Their Role in the Research Ecosystem. \textit{PLoS Biol} 11(1): e1001468. \url{doi:10.1371/journal.pbio.1001468} \\
  
\noindent{\sc Stephen A. Goff, Matthew Vaughn, Sheldon McKay, et al.} (2011) The iPlant Collaborative: Cyberinfrastructure for Plant Biology. \textit{Frontiers in PLant Genetics and Genomics} 2. \url{doi:10.3389/fpls.2011.0003} \\

\noindent{\sc Margaret E K Evans, David J Hearn, Kathryn E Theiss, Karen Cranston, Kent E Holsinger, Michael J Donoghue} (2011) Extreme environments select for reproductive assurance: evidence from evening primroses (\textit{Oenothera}). \textit{The New Phytologist} 191(2), pp. 555�563.\\

\noindent{\sc Jetty S S Ammiraju, Chuanzhu Fan, Yeisoo Yu, et al.} (2010) Spatio-temporal patterns of genome evolution in allotetraploid species of the genus \textit{Oryza}. \textit{The Plant Journal} 63(3), pp. 430-442.\\

\noindent{\sc Karen A Cranston} (2010) Quantifying gene tree incongruence across varying phylogenetic depths. In L. Lacey Knowles, Laura S. Kubatko (Eds.) \emph{Estimating Species Trees: Practical and Theoretical Aspects}, Wiley-Blackwell.\\

\noindent{\sc Karen A. Cranston, Bonnie Hurwitz, Michael J. Sanderson, Doreen Ware, Rod A. Wing, Lincoln Stein} (2010) Phylogenomic analysis from deep BAC-end sequence libraries of rice. \textit{Systematic Botany} 35(3), pp. 512-523\\

\noindent{\sc Karen A Cranston, Bonnie Hurwitz, Doreen Ware, Lincoln Stein, Rod A Wing} (2009) Species trees from highly incongruent gene trees in rice. \textit{Systematic Biology} 58 (5), 489-500 \url{doi: 10.1093/sysbio/syp054}\\

\noindent{\sc Michael J Sanderson, Darren Boss, Duhong Chen, Karen A Cranston, Andre Wehe} (2008) The PhyLoTA Browser: Processing GenBank for molecular phylogenetics research. \emph{Systematic Biology} 57(3), pp. 335-346.\\

\noindent{\sc Karen A Cranston, Bruce Rannala} (2007). Summarizing a posterior distribution of phylogenies using agreement subtrees. \emph{Systematic Biology} 56(4), pp. 578-590.\\

\noindent{\sc Karen A Cranston, Bruce Rannala} (2005). Molecular clocks: Closing the gap between rocks and clocks. \emph{Heredity} 94(5), pp. 461-462.
%\vskip 6mm

%In-prep
%\section*{Manuscripts Submitted}

% Invited talks
%\section*{Invited talks}
%\begin{itemize}
%\item{``Open Tree of Life: Synthesizing phylogenetic data into a tree of all life'', University of North Carolina Charlotte (2013)}
%\item{``Phylogenetics, phylogenomics and phyloinformatics: Finding evolutionary signals in large data sets'', University of Toronto (2010)}
%\item{``Automating phylogenetic tree inference from public sequence databases'', University of Missouri, 2009} 
%\item{``Automating phylogenetic tree inference from public sequence databases'', Botany meetings, 2009} 
%\item{``Species trees and gene trees from high-throughput sequence data in rice'', University of Michigan, 2009}
%\item{``A species tree from 2000 gene trees in rice (\textit{Oryza})'', Evolution, 2008}
%\item{``The PhyLoTA browser: Processing GenBank for molecular phylogenetic research'', George Washington University, 2008}
%\item{``Species trees from gene trees in rice: When 2.5 million characters are not enough...'', University of California Davis, 2008}
%\end{itemize}
%\vskip 4mm

% Conferences
%\section*{Conference presentations}

%\begin{itemize}
%\item{``Open Tree of Life: An automated and Community Assembled tree of life'', Evolution (2012)}
%\item{``Making the tree of life Phylotastic'', iEvoBio (2012)}
%\item{``Visualizing Phylogenies and Metadata at the Scale of One Million Tips'', iEvoBio (2010)}
%\item{``Phylogenetic inference strategies for thousands of gene trees'', Evolution (2010)}
%\item{``The PhyLoTA browser: Hierarchies of phylogenetic content from GenBank'', Evolution (2009)}
%\item{``Rice as a model system for phylogenomics'', Society for Molecular Biology and Evolution (2008)}
%\item{``Summarizing a posterior distribution of phylogenies'', Society for Molecular Biology and Evolution (2006)}
%\item{``Detecting convergence in Bayesian phylogenetic inference'', Evolution (2004)}
%\item{``A novel proposal mechanism for Bayesian phylogenetic inference'', Society for Molecular Biology and Evolution (2003)}
%\end{itemize}
%\vskip 4mm

%\newpage
% awards
\section*{Grants and Awards}
\begin{itemize}
\item{Collaborative Research: Automated and community-driven synthesis of the tree of life. Lead PI Cranston. NSF AVAToL (2011). \$5.76 million (\$927,000 to Duke)}
\item{The iPlant Collaborative; a cyberinfrastructure-centered community for a new plant biology - subcontract to Field Museum of Natural History, (\$62000, PI Cranston, 2009-2010) and to Duke University (\$14500, PI Vision, 2011-2011)}
\item{Dissertation Fellowship, University of Alberta (2006)}
\item{Province of Alberta Graduate Student Fellowship (2006)}
\item{Mary Louise Imrie Graduate Student Award, University of Alberta (2006)}
\item{75th Anniversary Graduate Student Award, Faculty of Medicine and Dentistry, University of Alberta (2004)}
\item{Medical Sciences Graduate Research Assistantship, Faculty of Medicine and Dentistry, University of Alberta (2006 \& 2004)}
\end{itemize}
%\vskip 4mm

% Other 'lookin good' stuff
\section*{Professional activities}
Many of these activities involve community building across biologists and programmers evolutionary informatics, through hackathons, summer internships, conferences and workshops. 
\begin{itemize}
\item{Workshop on Sustainable Software for Science: Practice and Experiences (WSSSPE), \url{http://wssspe.researchcomputing.org.uk/cfp/}, Organizing committee in 2014, Program committee in 2013}
\item{Informatics for Evolutionary Biology (iEvoBio) conference, Chair in 2014, Leadership team in 2013)}
\item{Organization administrator for NESCent Google Summer of Code participation, 2011-2013}
\item{Society of Systematic Biology council, 2012-present}
\item{Systematic Biology Editorial Board, 2011-present}
\item{Leadership team, Hackathons, Interoperability and Phylogenetics (HIP) NESCent working group, 2011-present}
\item{Phyloinformatics Research Foundation board member, 2010-present}
\item{Team lead, iPlant Tree Of Life working group on Tree Visualization, 2009-2012}
\item{Co-organizer: Phyloinformatics VoCamp for development of an ontology for evolutionary biology, November 2009, Montpellier, France}
\item{Invited participant: National Evolutionary Synthesis Center (NESCent) Hackathon on Evolutionary Database Interoperability, March 2009, Durham, NC, USA}
\item{Invited participant: iPlant Collaborative (\url{http://iplantcollaborative.org}) Grand Challenge Workshop on Assembling the Tree of Life to Enable the Plant Sciences, November 2009, Biosphere 2, Oracle, AZ}
\item{Journal referee for: Systematic Biology, Molecular Biology and Evolution, Bioinformatics, BMC Evolutionary Biology}
\item{National Science Foundation panelist, 2011, 2012, 2013}
%\item{Reviews for ``Computational Molecular Evolution" by Z. Yang, 2007}
%\item{Professional Memberships: Society of Systematic Biology, Society for Molecular Biology and Evolution, Botanical Society of America, Biodiversity Information Standards (TDWG)}
\end{itemize}

% Teaching
\section*{Teaching}
\begin{itemize}
\item{Coordinator, NESCent Academy (\url{http://academy.nescent.org}): a series of hands-on summer workshops in Evolutionary Biology and Informatics for graduate students, postdoctoral fellow and faculty, 2010-present.}
\item{Instructor: Software Carpentry and Data Carpentry (\url{http://software-carpentry.org}), 2012-present}
\item{Instructor: Computational biology workshops, Wuhan Institute of Virology and Fudan University, China, 2013}
\item{Instructor: Phylogenetics workshops, Kenya Medical Research Institute, 2012}
\item{Instructor: Phylogenetics workshop, Research Experience for Undergraduates (REU) Program, Field Museum of Natural History, 2010}
\item{Instructor: Workshop in applied phylogenetics, Bodega Bay Marine Laboratory, 2006}
\end{itemize}
%\vskip 4mm

\end{document}  
