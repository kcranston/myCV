\documentclass[10pt]{article}
\usepackage{geometry}                % See geometry.pdf to learn the layout options. There are lots.
\geometry{letterpaper}                   % ... or a4paper or a5paper or ... 
\usepackage{graphicx}
\usepackage[hyphens]{url}
\usepackage{fancyhdr}
\usepackage{hyperref}
\usepackage{parskip}
\usepackage{helvet}
\usepackage[all]{nowidow}

\renewcommand{\familydefault}{\sfdefault}
\renewcommand{\arraystretch}{2}
\hypersetup{backref,  urlcolor=blue, colorlinks=true, citecolor=black, linkcolor=black, hyperindex=true}
\setlength{\parskip}{3pt}

\title{Brief Article}
\author{The Author}
%\date{}                                           % Activate to display a given date or no date
\pagestyle{fancy}
\lhead{\bf \large Karen A. Cranston} 

\begin{document}
\noindent Biological Informatics Centre of Excellence \hfill \url{https://twitter.com/kcranstn} \\
Agriculture and Agri-food Canada  \hfill \url{http://kcranston.github.io} \\
Ottawa, ON \hfill  \url{http://github.com/kcranston} \\
karen.cranston@gmail.com  \hfill \url{http://orcid.org/0000-0002-4798-9499}

\vspace{0.4cm}
I am an evolutionary computational biologist who is passionate about building computational tools and teaching skills to facilitate the adoption and practice of more efficient, reproducible science. As both project manager at a research facility and lead PI of Open Tree of Life (\url{http://opentreeoflife.org}), I have been both developer and manager of complex, multi-faceted software projects in evolutionary informatics. I am deeply involved as a teacher and leader in efforts to teach computation skills to scientists through The Carpentries (\url{https://carpetries.org}). 

\section*{Skill summary}

\emph{Technical}: extensive experience with python (web applications, libraries, scripting); knowledge of database (SQL) design, API design / implementation, UI design, Ansible deployment; previous experience with C / C++, Perl \\

\emph{Outreach / education}: teaching interactive workshops, developing course material, coordinating training programs, and evidence-based instructor training; teaching technical and analytical skills to researchers; organization of hackathons for building tools and communities \\ 

\emph{Management}: project and financial management of scientific research grants, supervising technical staff across multiple projects, agile software development, board member overseeing open source / open science organizations   \\

\emph{Research}: phylogenetic methods, including gene tree inference, visualization, supertrees; statistical genetics, particularly Bayesian Markov Chain Monte Carlo \\

\section*{Education}
{\bf PhD} (2007) Statistical Genetics. Department of Medical Genetics, University of Alberta, Edmonton, AB, Canada. 

{\bf B.Sc., Honours, First Class} (1996), Genetics. University of Manitoba, Winnipeg, MB, Canada \\

% Education
\section*{Employment}

\noindent \textbf{Technical Advisor, Bioinformatics} \\
2017-present, Agriculture and Agri-food Canada, Ottawa, ON \\
Project manager and technical advisor in biological informatics group, leading development of software for biological specimen collection digitization and management.\\

\noindent \textbf{Research Scientist} \\
2015-2017, Department of Biology, Duke University, Durham, NC \\
Lead PI of Open Tree of Life (\url{http://opentreeoflife.org}) project that summarizes published phylogenetic trees into a synthetic tree of life; designing and writing software for storing, accessing and analyzing phylogeny data; leading hackathons and workshops; managing overall project, including software development. \\

\noindent \textbf{Training Coordinator and Bioinformatics Project Manager} \\
2010-2015, National Evolutionary Synthesis Center (NESCent), Durham, NC. \\
Management of informatics support for a broad range of evolutionary science projects; coordination of training programs in informatics and analytical skills for scientists; community building in open-source biodiversity and evolutionary informatics through hackathons and Google Summer of Code. \\

\noindent \textbf{Postdoctoral Research Associate} \\
2009-2010, Biodiversity Synthesis Center, The Field Museum, Chicago, IL  \\
Interactive visualization of taxonomies and large phylogenies with Encylopedia of Life and iPlant Collaborative; summarizing large-scale gene tree incongruence across genomes \\

\noindent \textbf{Postdoctoral Research Associate} \\
2007-2009, Department of Ecology and Evolutionary Biology, University of Arizona, Tucson, AZ \\
PhyLoTA broswer, summarizing phylogenetic signal in GenBank; phylogenomic analyses of wild and domesticated rice species \\ 

\noindent \textbf{Laboratory technician} \\
1997-2002, Cangene Corporation, Winnipeg, MB. \\
Maintenance of cell cultures; design and implementation of protein chemistry assays; analytical method validation. \\

\noindent \textbf{Technical writer}
1996-1997, Integrated Engineering Software, Winnipeg, Canada, 1996-1997 \\
Technical documentation and online help systems for engineering simulation software. \\

%Pubs
%\vspace{2mm}
\section*{Publications}

I have 19 peer-reviewed scientific papers, and 6 commentaries / preprints (noted with *). Only papers from 2013-2018 shown below. Full list at \url{http://kcranston.github.io/research/}.

\subsection*{Phylogenetic methods}
McTavish, E. J., Drew, B. T., Redelings, B., \& Cranston, K. A. (2017). How and Why to Build a Unified Tree of Life. BioEssays, 39(11). \url{https://doi.org/10.1002/bies.201700114}

Rees, J., \& Cranston, K. (2017). Automated assembly of a reference taxonomy for phylogenetic data synthesis. Biodiversity Data Journal, 5, e12581. \url{https://doi.org/10.3897/BDJ.5.e12581}

Hinchliff, C. E., et al., (2015). Synthesis of phylogeny and taxonomy into a comprehensive tree of life. Proceedings of the National Academy of Sciences, 112(41), 12764-12769. \url{http://dx.doi.org/10.1073/pnas.1423041112}. 

\subsection*{Software and databases}
McTavish, E. J., et al., (2015). Phylesystem: a git-based data store for community curated phylogenetic estimates. Bioinformatics, btv276. \url{http://dx.doi.org/10.1093/bioinformatics/btv276} 

Ksepka, D.T., et al., (2015) The Fossil Calibration Database, A New Resource for Divergence Dating. \textit{Systematic Biology}. \url{http://dx.doi.org/10.1093/sysbio/syv025}

Stoltzfus, A., et al., (2013). Phylotastic! Making tree-of-life knowledge accessible, reusable and convenient. \textit{BMC bioinformatics}, 14(1), 158. \url{http://dx.doi.org/10.1186/1471-2105-14-158} 

\subsection*{Open science, Collaboration, and Education}
Wilson, G., et al., (2017). Good Enough Practices in Scientific Computing. PLoS Comput Biol 13(6): e1005510 \url{https://doi.org/10.1371/journal.pcbi.1005510}

Katz, D.S. et al., (2016). Report on the Second Workshop on Sustainable Software for Science: Practice and Experiences (WSSSPE2). Journal of Open Research Software. 4(1), p.e7. \url{http://doi.org/10.5334/jors.85}

Teal, T. K., et al., (2015) Data Carpentry: Workshops to Increase Data Literacy for Researchers. \textit{International Journal of Digital Curation}. 10(1):135-143. \url{http://dx.doi.org/10.2218/ijdc.v10i1.351}
 
Cranston, K. A., et al., (2014) Best Practices for Data Sharing in Phylogenetic Research. \textit{PLOS Currents Tree of Life}. Jun 19. Edition 1. \\
\url{http://dx.doi.org/10.1371/currents.tol.bf01eff4a6b60ca4825c69293dc59645} 

* Katz, D. S., et al., (2014). Second Workshop on on Sustainable Software for Science: Practice and Experiences (WSSSPE2): Submission, Peer-Review and Sorting Process, and Results. arXiv preprint \url{http://arxiv.org/abs/1411.3464}.

* Vision, T., \& Cranston, K. A. (2014) Open data for evolutionary synthesis: an introduction to the NESCent collection. Nature Scientific Data. \url{http://dx.doi.org/10.1038/sdata.2014.30}

* Cranston, K. A., et al., (2014): Simple rules for sharing phylogenetic data. figshare. \url{http://dx.doi.org/10.6084/m9.figshare.997763}

Rodrigo, A., et al., (2013). Science incubators: synthesis centers and their role in the research ecosystem. PloS biology, 11(1), e1001468. \url{http://dx.doi.org/10.1371/journal.pbio.1001468} 

* Rees, J. A., et al., (2013): Response to GBIF request for consultation on data licenses. figshare. \url{http://dx.doi.org/10.6084/m9.figshare.799766}

* Cranston, K. A., et al., (2013): A grassroots approach to software sustainability. figshare.
\url{http://dx.doi.org/10.6084/m9.figshare.790739}

\section*{Education and outreach}

\begin{itemize}
\item{NESCent Academy: coordinated series of hands-on summer workshops in Evolutionary Biology and Informatics, 2010-2014.}
\item{Software / Data Carpentry: Executive Council Chair since 2018; instructor certification in 2013; Instructor Training certification in 2017; co-lead development of reproducible science curriculum}
\item{Phylogenetics workshops: taught phylogenetic methods and software in workshops at Wuhan Institute of Virology, Fudan University, Kenyan Medical Research Centre, The Field Museum, and Bodega Bay Marine Laboratory} 
\item{Hackathons: co-organized four hackathons on evolutionary informatics through NESCent and Open Tree of Life; see \url{https://informatics.nescent.org/wiki/Main_Page}}
\end{itemize}

% Other 'lookin good' stuff
\section*{Professional activities}
\begin{itemize}
\item{Board of Directors: Open Bioinformatics Foundation (\url{https://open-bio.org/}), 2015-present}
\item{The Carpentries Chair of Executive Council (\url{http://carpentries.org}), 2017-present}
\item{Data Carpentry founding board member (\url{http://datacarpentry.org}), 2014-2017}
\item{Organization administrator for NESCent Google Summer of Code participation, 2011-2013}
\item{Program committee / panelist, Workshop on Sustainable Software for Science: Practice and Experiences (WSSSPE), \url{http://wssspe.researchcomputing.org.uk}, 2013, 2014, 2016}
\item{Society of Systematic Biology council, 2012-2014}
\item{Systematic Biology Editorial Board, 2011-present}
\item{Leadership team, Informatics for Evolutionary Biology (iEvoBio, \url{http://ievobio.org}) conference, 2013-2016}
\item{Phyloinformatics Research Foundation board member, 2010-2016}
\item{Team lead, iPlant Tree Of Life working group on Tree Visualization, 2009-2012}
\item{Invited participant: iPlant Collaborative (\url{http://iplantcollaborative.org}) Grand Challenge Workshop on Assembling the Tree of Life to Enable the Plant Sciences, November 2009, Biosphere 2, Oracle, AZ}
\item{Journal referee for: Systematic Biology, Trends Ecology and Evolution, PeerJ, Nucleic Acids Research, Journal of Open Research Software, PLOS Computational Biology, Molecular Biology and Evolution}. See \href{https://publons.com/a/213683/}{Publons reviewer profile} for recent activity. 
\item{National Science Foundation grant review panelist, 2011 - 2014}
\end{itemize}

\end{document}  
